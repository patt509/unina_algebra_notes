\chapter{Introduzione}
\introduction{G. Cutolo}{}

\newpage

\section{Sulle versioni del documento}

\warning{Attento}{
	\textbf{Esistono diverse versioni di questo documento in circolazione}. È fondamentale assicurarsi di \textbf{consultare sempre la versione più recente}, in quanto potrebbe contenere informazioni aggiornate o correzioni rispetto alle versioni precedenti. Per evitare di studiare da fonti non esatte si raccomanda vivamente di verificare la data di pubblicazione e il numero di revisione riportati alla fine di questa pagina.
}

\begin{minipage}{.45\textwidth}
	\begin{center}
		\begin{tblr}{
			colspec = {c|c},
			hlines,
			vlines,
			row{1} = {font=\bfseries, primary!80},
			width = \linewidth
		}
		Revisione & Data \\
		\SetRow{red9} 67dfc7c & 15/03/2024 \\
		\SetRow{red9} 00a8221 & 28/08/2024 \\
		\SetRow{red9} 343cc71 & 29/08/2024  \\
		\SetRow{red9}9087dcf & 11/09/2024  \\
		\SetRow{yellow9} 83a3f0f & 01/10/2024 \\
		\SetRow{yellow9} 3af1354 & 19/02/2025 \\
		\SetRow{yellow9}  3d59ec1 & 05/09/2025 \\
		\SetRow{green9}  6a13asd & 04/10/2025
	\end{tblr}
	\captionof{table}{Cronologia revisioni del documento}
	\end{center}
\end{minipage}
\hfil
\begin{minipage}{.45\textwidth}
	\begin{center}
		\includegraphics[scale=.35]{sources/Semaforo.drawio}
	\end{center}
\end{minipage}


\section{Repository del progetto}
Tutte le versioni del documento, insieme al codice sorgente e ai materiali aggiuntivi, sono disponibili nella \href{https://github.com/Giordi9902/unina_algebra_notes}{repository} ufficiale su GitHub.

Ti invitiamo a consultare la repository per eventuali aggiornamenti, contributi o per segnalare problemi direttamente tramite issue.

\subsection{In caso di errori}
È sempre ben gradito ricevere feedback.

\textbf{Feedback generali:} se hai domande rispetto qualsiasi aspetto del libro non esitare a contattarci:
\begin{itemize}
	\item Email: \texttt{giorgio99difusco@gmail.com} (Creatore delle dispense)
	\item Telegram: \texttt{@Iracondia} (Manutentore)
\end{itemize}
